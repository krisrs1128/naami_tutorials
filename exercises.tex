\documentclass{article}
\usepackage{graphicx,amsmath,amssymb}
% ------------------------------------------------------------------------
% Packages
% ------------------------------------------------------------------------
\usepackage{amsmath}
\usepackage{tabularx}

% ------------------------------------------------------------------------
% Macros
% ------------------------------------------------------------------------
%~~~~~~~~~~~~~~~
% List shorthand
%~~~~~~~~~~~~~~~
\newcommand{\BIT}{\begin{itemize}}
\newcommand{\EIT}{\end{itemize}}
\newcommand{\BNUM}{\begin{enumerate}}
\newcommand{\ENUM}{\end{enumerate}}
%~~~~~~~~~~~~~~~
% Text with quads around it
%~~~~~~~~~~~~~~~
\newcommand{\qtext}[1]{\quad\text{#1}\quad}
%~~~~~~~~~~~~~~~
% Shorthand for math formatting
%~~~~~~~~~~~~~~~
\newcommand\mbb[1]{\mathbb{#1}}
\newcommand\mbf[1]{\mathbf{#1}}
\def\mc#1{\mathcal{#1}}
\def\mrm#1{\mathrm{#1}}
%~~~~~~~~~~~~~~~
% Common sets
%~~~~~~~~~~~~~~~
\def\reals{\mathbb{R}} % Real number symbol
\def\integers{\mathbb{Z}} % Integer symbol
\def\rationals{\mathbb{Q}} % Rational numbers
\def\naturals{\mathbb{N}} % Natural numbers
\def\complex{\mathbb{C}} % Complex numbers
\def\simplex{\mathcal{S}} % Simplex
%~~~~~~~~~~~~~~~
% Common functions
%~~~~~~~~~~~~~~~
\renewcommand{\exp}[1]{\operatorname{exp}\left(#1\right)} % Exponential
\def\indic#1{\mbb{I}\left({#1}\right)} % Indicator function
\providecommand{\maximize}{\mathop\mathrm{maximize}} % Defining math symbols
\providecommand{\minimize}{\mathop\mathrm{minimize}}
\providecommand{\argmax}{\mathop\mathrm{arg max}}
\providecommand{\argmin}{\mathop\mathrm{arg min}}
\providecommand{\arccos}{\mathop\mathrm{arccos}}
\providecommand{\asinh}{\mathop\mathrm{asinh}}
\providecommand{\dom}{\mathop\mathrm{dom}} % Domain
\providecommand{\range}{\mathop\mathrm{range}} % Range
\providecommand{\diag}{\mathop\mathrm{diag}}
\providecommand{\tr}{\mathop\mathrm{tr}}
\providecommand{\abs}{\mathop\mathrm{abs}}
\providecommand{\card}{\mathop\mathrm{card}}
\providecommand{\sign}{\mathop\mathrm{sign}}
\def\rank#1{\mathrm{rank}({#1})}
\def\supp#1{\mathrm{supp}({#1})}
%~~~~~~~~~~~~~~~
% Common probability symbols
%~~~~~~~~~~~~~~~
\def\E{\mathbb{E}} % Expectation symbol
\def\Earg#1{\E\left[{#1}\right]}
\def\Esubarg#1#2{\E_{#1}\left[{#2}\right]}
\def\P{\mathbb{P}} % Probability symbol
\def\Parg#1{\P\left({#1}\right)}
\def\Psubarg#1#2{\P_{#1}\left[{#2}\right]}
\def\Cov{\mrm{Cov}} % Covariance symbol
\def\Corr{\mrm{Corr}} % Covariance symbol
\def\Covarg#1{\Cov\left[{#1}\right]}
\def\Covsubarg#1#2{\Cov_{#1}\left[{#2}\right]}
\def\Corrsubarg#1#2{\Corr_{#1}\left[{#2}\right]}
\def\Var{\mrm{Var}}
\def\Vararg#1{\Var\left(#1\right)}
\def\Varsubarg#1#2{\Var_{#1}\left(#2\right)}
\newcommand{\family}{\mathcal{P}} % probability family
\newcommand{\eps}{\epsilon}
\def\absarg#1{\left|#1\right|}
\def\msarg#1{\left(#1\right)^{2}}
\def\logarg#1{\log\left(#1\right)}
%~~~~~~~~~~~~~~~
% Distributions
%~~~~~~~~~~~~~~~
\def\Gsn{\mathcal{N}}
\def\Ber{\textnormal{Ber}}
\def\Bin{\textnormal{Bin}}
\def\Unif{\textnormal{Unif}}
\def\Mult{\textnormal{Mult}}
\def\Cat{\textnormal{Cat}}
\def\Gam{\textnormal{Gam}}
\def\InvGam{\textnormal{InvGam}}
\def\NegMult{\textnormal{NegMult}}
\def\Dir{\textnormal{Dir}}
\def\Lap{\textnormal{Laplace}}
\def\Bet{\textnormal{Beta}}
\def\Poi{\textnormal{Poi}}
\def\HypGeo{\textnormal{HypGeo}}
\def\GEM{\textnormal{GEM}}
\def\BP{\textnormal{BP}}
\def\DP{\textnormal{DP}}
\def\BeP{\textnormal{BeP}}
%~~~~~~~~~~~~~~~
% Theorem-like environments
%~~~~~~~~~~~~~~~

%-----------------------
% Probability sets
%-----------------------
\newcommand{\X}{\mathcal{X}}
\newcommand{\Y}{\mathcal{Y}}
\newcommand{\D}{\mathcal{D}}
\newcommand{\Scal}{\mathcal{S}}
%-----------------------
% vector notation
%-----------------------
\newcommand{\bx}{\mathbf{x}}
\newcommand{\by}{\mathbf{y}}
\newcommand{\bt}{\mathbf{t}}
\newcommand{\xbar}{\overline{x}}
\newcommand{\Xbar}{\overline{X}}
\newcommand{\tolaw}{\xrightarrow{\mathcal{L}}}
\newcommand{\toprob}{\xrightarrow{\mathbb{P}}}
\newcommand{\laweq}{\overset{\mathcal{L}}{=}}
\newcommand{\F}{\mathcal{F}}


\title{Exercises for Advanced Deep Learning}

\begin{document}

\section{Interpretability}

\begin{enumerate}
\item Visualizing learned features using t-SNE. t-distributed Stochastic
  Neighbor Embedding (t-SNE) is a method for visualizing high dimensional data
  in low-dimensional space, such that points close toghether in the
  high-dimensional space remain close in the low-dimensional visualization.
  Propose a method for visualizing features learned at various depth in a deep
  learning model, and then compare your approach to the one described at
  \url{https://cs.stanford.edu/people/karpathy/cnnembed/}.

  \item Testing with CAVs. There is a risk when working with CAVs that you learn
    a totally meaningless concept -- the procedure returns a CAV even if you
    defined a concept using totally random features. Define a statistic based on
    the CAV scores for class $k$ and layer $l$ by
    \begin{align*}
      \frac{\#\{S_{C, k, l}\left(x_i\right) > 0\}}{\#\{\text{examples in class } k\}}
    \end{align*}
    which measures the fraction of samples in class $k$ which are positively
    activated by the given concept. Propose a statistical test for finding out
    whether this fraction is meaningfully large; i.e., that it is larger than
    you would have if you had used totally random images to define a (totally
    meaningless) concept.
\end{enumerate}

\section{GANs}

\begin{enumerate}
\item Using the figure on the first formulation slide, come up with a visual
  interpretation of the change of variables formula, which says that if $x
  \xrightarrow{f} y$ and if $x \sim p\left(x\right)$, then $y \sim
  p\left(f^{-1}\left(y\right)\right)\absarg{\frac{df}{dx}}^{-1}$.
\item GANs and VAEs are both generative models in the sense that you can sample
  new data from them. One however allows you to sample latent encodings $z$ for
  any $x$ of interest, and the other does not, which is which?
\item Verify the density ratio estimation claim from the lecture that
  $\frac{p^{\ast}\left(x \vert y = 1\right)}{q_{\theta}\left(x\right)} =
  \frac{p\left(y = 1\vert x\right)}{p\left(y = 0\right)}\frac{1 - \pi}{\pi}$.
  Hint: Use Bayes' rule.
\item Instead of a completely unsupervised GAN, you can learn to generate
  samples conditional on a class label $y$. Explain why an objective like,

  \begin{align*}
    \min_{G}\max_{D} V\left(D, G\right) := \Esubarg{p_{\text{data}}}{\log D\left(x \vert y\right)} + \Esubarg{p\left(z\right)}{\log\left(1 - D\left(G\left(z \vert y\right)\right)\right)}
  \end{align*}
\end{enumerate}
might be able to work (see Mirza and Osindro for an actual implementation).

\section{Metalearning}

\begin{enumerate}
\item Identify some contexts where metalearning could be applied in practice.
  Are there limitations in the metalearning setup that make it less useful in
  scenarios you think of?

\item In transfer learning, you may choose to fine tune the lower layer weights
  on your new task, rather than simply copying the original features verbatim.
  If this is your goal, how should you choose your learning rates for the
  low-level features, versus the new high-level weights?

\item For $k$-nearest neighbors, larger $k$ reduces variance but increases bias
  -- it controls model complexity. In the nearest neighbors metalearner, we
  aren't using nearest neighbors direction, but some smoothed-out version of it.
  How might you control model complexity for this alternative version of nearest
  neighbors?

\item How would you adapt the ordinary classification-based nearest neighbors
  metalearner to work with continuous $y_i$ instead?

\end{enumerate}

\section{Bayesian Deep Learning}

\begin{enumerate}

\item Assignments in mixture of Gaussians. Suppose $x_i$ is drawn from a mixture
  of two gaussians, which have parameters $\left(\mu_{1}, \sigma_{1}^2\right) =
  \left(0, 1\right)$ and $\left(\mu_{2}, \sigma_{2}^{2}\right) = \left(2,
  1\right)$. Show that $p\left(z = 1\vert x = 1\right) = \frac{1}{2}$ and
  $p\left(z = 1 \vert x = 0\right) = \frac{1}{1 + \exp{-2}} \approx 0.881$. In
  general the posterior is Bernoulli with probability $\varphi\left(x\right)$ of
  assigning to class 1. Can you find a formula for $\varphi\left(x\right)$ that
  applies to general (or multivariate?) $\mu_{k}, \Sigma_{k}$?

%% \item Jensen's inequality. The proof that $D_{KL}$ is nonnegative hinges heavily
%%   on a fact called Jensen's inequality. We prove this inequality here.
%% \begin{itemize}
%% \item A convex function is one for which, for any $\lambda \left(0, 1\right)$,
%%   we have $f\left(\lambda x + \left(1 - \lambda\right) y\right) \leq \lambda
%%   f\left(x\right) + \left(1 - \lambda\right)f\left(y\right)$. Interpret this
%%   inequality geometrically. (hint: consider the case of $\lambda = \frac{1}{2}$.
%% \item Let $f\left(x\right) = x^2$, and suppose $X \sim \Unif\left(-1, 1\right)$.
%% \item Hence, argue that $D_{KL}\left(q \vert \vert p\right)$
%% \end{itemize}

\item More general reparameterization. We saw that the Gaussian distribution
$\Gsn\left(x \vert \mu, \sigma^2 I\right)$ can be reparameterized as $g_{\mu,
  \sigma}\left(x\right) = \mu + \sigma \bigodot \eps$, where $\eps \sim
\Gsn\left(\eps \vert 0, I\right)$ doesn't depend on any parameters. This trick
actually applies for a variety of other distributions, which this exercise
explores.
\begin{itemize}
\item Random Variable generation using inverse CDFs\footnote{Cumulative Distribution Functions}. Suppose
  that $Z$, which we assume is one-dimensional, has CDF $F\left(z\right)$. Let
  $U \sim \Unif\left(0, 1\right)$. Verify that the transformation of $u$ defined
  by $F^{-1}\left(U\right)$ has CDF $F\left(z\right)$, and so has the same
  distribution as $Z$.
\item Argue that whenever the CDF of the density $q_{\varphi}\left(z \vert
  x\right)$ is known, this allows for a version of the reparameterization trick.
\item Suppose that $Z \sim \Exp{\lambda}$, meaning that it has CDF function
  $F\left(z\right) = 1 - \exp{-\lambda z}$. How can you simulate this?
\item Can you think of downsides of this approach?
\end{itemize}

\item Amortization vs. Approximation gaps. Recall that in the derivation of the
  ELBO, we had an expression like
  \begin{align*}
    \log p_{\theta}\left(x\right) &= \Esubarg{q}{\log p_{\theta}\left(x \vert
      z\right)} - D_{KL}\left(q\left(z \vert x\right) \vert \vert
    p\left(z\right)\right) + D_{KL}\left(q\left(z \vert x\right) \vert \vert
    p\left(z \vert x\right)\right)
  \end{align*}
  and we dropped the last term from the optimization, because it is intractable.
  We now study the role of that term when proposing variational families.

  \begin{itemize}
  \item In the usual VAE, we set $q_{\varphi}\left(z \vert x\right) =
    \Gsn\left(z \vert \mu_{\varphi}\left(x\right),
    \sigma_{\varphi}^{2}\left(x\right)I\right)$; i.e., a diagonal gaussian..
    Suppose you had approximated it instead by a Gaussian with general
    $\Sigma\left(x\right)$. What effect would this have on
    $D_{KL}\left(q_{\varphi^{\ast}}\left(z \vert x\right) \vert \vert p\left(z
    \vert x\right)\right)$, when considering the best possible
    $q_{\varphi^\ast}$ from either of these two (diagonal or dense covariance)
    variational families?
  \item Let $\hat{\varphi}$ be the parameters of the fitted inference network
    after optimizing the ELBO. Express the final inference quality
    $D_{KL}\left(q_{\hat{\varphi}}\left(z \vert x\right) \vert \vert p\left(z
    \vert x\right)\right)$ as,
    \begin{align*}
      D_{KL}\left(q_{\varphi^{\ast}}\left(z \vert x\right) \vert \vert p\left(z
      \vert x\right)\right) + \left(D_{KL}\left(q_{\hat{\varphi}}\left(z \vert
      \vert x\right)\right) - D_{KL}\left(q_{\varphi^{\ast}}\left(z \vert x
        \right)\right)\right).
    \end{align*}
    The first term (outside parenthesis) is called the ``approximation gap,''
    and refers to the difference between the true posterior and the best
    possible element of the variational approximation, while the second term (in
    parenthesis) is called the ``amortization gap,'' and refers to the
    difference between the best possible approximation within the family and
    what is actually found by the network. In light of this discussion, why
    might we choose not to proceed with the full $\Sigma\left(x\right)$
    parameterization in the previous part?
  \item A normalizing flow is a sequence of transformations to a simple variable
    that results in a variable with a more complicated density, but one which
    can still be written in closed form, using the change of variables formula.
    For example, you might iteratively apply $f\left(z\right) = z +
    u\sigma\left(w^T z + b\right)$ to what is intiially a simple (say gaussian
    $z$), since this transformation is easy to differentiate (which is the only
    thing you need to apply the change of variables formula). Does this proposed
    procedure reduce the approximation or amortization gap?
  \item What are some general strategies for reducing the amortization gap?
  \end{itemize}
\end{enumerate}

\end{document}
