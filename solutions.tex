\documentclass{article}
\usepackage{graphicx,amsmath,amssymb}
% ------------------------------------------------------------------------
% Packages
% ------------------------------------------------------------------------
\usepackage{amsmath}
\usepackage{tabularx}

% ------------------------------------------------------------------------
% Macros
% ------------------------------------------------------------------------
%~~~~~~~~~~~~~~~
% List shorthand
%~~~~~~~~~~~~~~~
\newcommand{\BIT}{\begin{itemize}}
\newcommand{\EIT}{\end{itemize}}
\newcommand{\BNUM}{\begin{enumerate}}
\newcommand{\ENUM}{\end{enumerate}}
%~~~~~~~~~~~~~~~
% Text with quads around it
%~~~~~~~~~~~~~~~
\newcommand{\qtext}[1]{\quad\text{#1}\quad}
%~~~~~~~~~~~~~~~
% Shorthand for math formatting
%~~~~~~~~~~~~~~~
\newcommand\mbb[1]{\mathbb{#1}}
\newcommand\mbf[1]{\mathbf{#1}}
\def\mc#1{\mathcal{#1}}
\def\mrm#1{\mathrm{#1}}
%~~~~~~~~~~~~~~~
% Common sets
%~~~~~~~~~~~~~~~
\def\reals{\mathbb{R}} % Real number symbol
\def\integers{\mathbb{Z}} % Integer symbol
\def\rationals{\mathbb{Q}} % Rational numbers
\def\naturals{\mathbb{N}} % Natural numbers
\def\complex{\mathbb{C}} % Complex numbers
\def\simplex{\mathcal{S}} % Simplex
%~~~~~~~~~~~~~~~
% Common functions
%~~~~~~~~~~~~~~~
\renewcommand{\exp}[1]{\operatorname{exp}\left(#1\right)} % Exponential
\def\indic#1{\mbb{I}\left({#1}\right)} % Indicator function
\providecommand{\maximize}{\mathop\mathrm{maximize}} % Defining math symbols
\providecommand{\minimize}{\mathop\mathrm{minimize}}
\providecommand{\argmax}{\mathop\mathrm{arg max}}
\providecommand{\argmin}{\mathop\mathrm{arg min}}
\providecommand{\arccos}{\mathop\mathrm{arccos}}
\providecommand{\asinh}{\mathop\mathrm{asinh}}
\providecommand{\dom}{\mathop\mathrm{dom}} % Domain
\providecommand{\range}{\mathop\mathrm{range}} % Range
\providecommand{\diag}{\mathop\mathrm{diag}}
\providecommand{\tr}{\mathop\mathrm{tr}}
\providecommand{\abs}{\mathop\mathrm{abs}}
\providecommand{\card}{\mathop\mathrm{card}}
\providecommand{\sign}{\mathop\mathrm{sign}}
\def\rank#1{\mathrm{rank}({#1})}
\def\supp#1{\mathrm{supp}({#1})}
%~~~~~~~~~~~~~~~
% Common probability symbols
%~~~~~~~~~~~~~~~
\def\E{\mathbb{E}} % Expectation symbol
\def\Earg#1{\E\left[{#1}\right]}
\def\Esubarg#1#2{\E_{#1}\left[{#2}\right]}
\def\P{\mathbb{P}} % Probability symbol
\def\Parg#1{\P\left({#1}\right)}
\def\Psubarg#1#2{\P_{#1}\left[{#2}\right]}
\def\Cov{\mrm{Cov}} % Covariance symbol
\def\Corr{\mrm{Corr}} % Covariance symbol
\def\Covarg#1{\Cov\left[{#1}\right]}
\def\Covsubarg#1#2{\Cov_{#1}\left[{#2}\right]}
\def\Corrsubarg#1#2{\Corr_{#1}\left[{#2}\right]}
\def\Var{\mrm{Var}}
\def\Vararg#1{\Var\left(#1\right)}
\def\Varsubarg#1#2{\Var_{#1}\left(#2\right)}
\newcommand{\family}{\mathcal{P}} % probability family
\newcommand{\eps}{\epsilon}
\def\absarg#1{\left|#1\right|}
\def\msarg#1{\left(#1\right)^{2}}
\def\logarg#1{\log\left(#1\right)}
%~~~~~~~~~~~~~~~
% Distributions
%~~~~~~~~~~~~~~~
\def\Gsn{\mathcal{N}}
\def\Ber{\textnormal{Ber}}
\def\Bin{\textnormal{Bin}}
\def\Unif{\textnormal{Unif}}
\def\Mult{\textnormal{Mult}}
\def\Cat{\textnormal{Cat}}
\def\Gam{\textnormal{Gam}}
\def\InvGam{\textnormal{InvGam}}
\def\NegMult{\textnormal{NegMult}}
\def\Dir{\textnormal{Dir}}
\def\Lap{\textnormal{Laplace}}
\def\Bet{\textnormal{Beta}}
\def\Poi{\textnormal{Poi}}
\def\HypGeo{\textnormal{HypGeo}}
\def\GEM{\textnormal{GEM}}
\def\BP{\textnormal{BP}}
\def\DP{\textnormal{DP}}
\def\BeP{\textnormal{BeP}}
%~~~~~~~~~~~~~~~
% Theorem-like environments
%~~~~~~~~~~~~~~~

%-----------------------
% Probability sets
%-----------------------
\newcommand{\X}{\mathcal{X}}
\newcommand{\Y}{\mathcal{Y}}
\newcommand{\D}{\mathcal{D}}
\newcommand{\Scal}{\mathcal{S}}
%-----------------------
% vector notation
%-----------------------
\newcommand{\bx}{\mathbf{x}}
\newcommand{\by}{\mathbf{y}}
\newcommand{\bt}{\mathbf{t}}
\newcommand{\xbar}{\overline{x}}
\newcommand{\Xbar}{\overline{X}}
\newcommand{\tolaw}{\xrightarrow{\mathcal{L}}}
\newcommand{\toprob}{\xrightarrow{\mathbb{P}}}
\newcommand{\laweq}{\overset{\mathcal{L}}{=}}
\newcommand{\F}{\mathcal{F}}


\title{Exercises for Advanced Deep Learning}

\begin{document}

\section{Interpretability}

\begin{enumerate}
\item Visualizing learned features using t-SNE. t-distributed Stochastic
  Neighbor Embedding (t-SNE) is a method for visualizing high dimensional data
  in low-dimensional space, such that points close toghether in the
  high-dimensional space remain close in the low-dimensional visualization.
  Propose a method for visualizing features learned at various depth in a deep
  learning model, and then compare your approach to the one described at
  \url{https://cs.stanford.edu/people/karpathy/cnnembed/}.

  A: The idea is to run $t$-SNE on the activations (from a layer of interest) of
  some example images. This will place iamges that are closer together in the
  representational space close to one another. You can then plot the original
  images at coordinates defined by the $t$-SNE. A neat thing you see is that for
  higher layers, close by images are semantically close but not necessarily
  pixel-wise close.

  \item Testing with CAVs. There is a risk when working with CAVs that you learn
    a totally meaningless concept -- the procedure returns a CAV even if you
    defined a concept using totally random features. Define a statistic based on
    the CAV scores for class $k$ and layer $l$ by
    \begin{align*}
      \frac{\#\{S_{C, k, l}\left(x_i\right) > 0\}}{\#\{\text{examples in class } k\}}
    \end{align*}
    which measures the fraction of samples in class $k$ which are positively
    activated by the given concept. Propose a statistical test for finding out
    whether this fraction is meaningfully large; i.e., that it is larger than
    you would have if you had used totally random images to define a (totally
    meaningless) concept.

    A: This has the flavor of a randomization test in statistics, where you
    learn a null distribution for a test statistic by sampling from it. In this
    case, you could sample from random images many times, to learn a ``random
    $S_{C,k.l}$ distribution. To get some sense of the meaningfulness of a CAV,
    just compare this score with this null distribution.
\end{enumerate}

\section{GANs}

\begin{enumerate}
\item Using the figure on the first formulation slide, come up with a visual
  interpretation of the change of variables formula, which says that if $x
  \xrightarrow{f} y$ and if $x \sim p\left(x\right)$, then $y \sim
  p\left(f^{-1}\left(y\right)\right)\absarg{\frac{df}{dx}}^{-1}$.

  A: The $p\left(f^{-1}\left(y\right)\right)$ term just references the original
  variables' density, by mapping back to the $x$-space. The interesting point
  visually is that if you have a flat part in the forwards mapping, a lot of the
  original probability density gets collapsed into that range of $y$'s. But
  these flat regions are exactly those for which $\absarg{\frac{df}{dx}}$ is
  small (and so it's inverse is large, and we put more probability mass there).

\item GANs and VAEs are both generative models in the sense that you can sample
  new data from them. One however allows you to sample latent encodings $z$ for
  any $x$ of interest, and the other does not, which is which?

  A: GANs can only be used to sample $z \vert x$, though there are proposals for
  combining VAEs with GANs in various ways (for example, ``Adversarially Learned
  Inference'')

\item Verify the density ratio estimation claim from the lecture that
  $\frac{p^{\ast}\left(x \vert y = 1\right)}{q_{\theta}\left(x\right)} =
  \frac{p\left(y = 1\vert x\right)}{p\left(y = 0\right)}\frac{1 - \pi}{\pi}$.
  Hint: Use Bayes' rule.

  A:
  \begin{align*}
     \frac{p^{\ast}\left(x\right)}{q_{\theta}\left(x\right)} &= \frac{p\left(y =
       1 \vert x\right)p\left(x\right)}{p\left(y = 1\right)} \frac{p\left(y =
       0\right)}{p\left(y = 0 \vert x\right)p\left(x\right)} \\
     &= \frac{p\left(y = 1 \vert x\right)}{p\left(y = 0 \vert x\right)} \frac{1 - \pi}{\pi}
  \end{align*}

\item Instead of a completely unsupervised GAN, you can learn to generate
  samples conditional on a class label $y$. Explain why an objective like,
  \begin{align*}
    \min_{G}\max_{D} V\left(D, G\right) := \Esubarg{p_{\text{data}}}{\log D\left(x \vert y\right)} + \Esubarg{p\left(z\right)}{\log\left(1 - D\left(G\left(z \vert y\right)\right)\right)}
  \end{align*}
\end{enumerate}
might be able to work, and propose an architecture that could be used (see Mirza
and Osindro for an actual implementation).

A: The point is that you are both generating and discriminating conditional on
class labels. In practice, this is usually done by concatenating the class label
(or an embedding of the class label) into the first layers of both the
discriminator and generator networks.

\section{Metalearning}

\begin{enumerate}
\item Identify some contexts where metalearning could be applied in practice.
  Are there limitations in the metalearning setup that make it less useful in
  scenarios you think of?

  A: ? I presented the ones I could think of in the lecture.

\item In transfer learning, you may choose to fine tune the lower layer weights
  on your new task, rather than simply copying the original features verbatim.
  If this is your goal, how should you choose your learning rates for the
  low-level features, versus the new high-level weights?

  A: You should use lower learning rates for the low-level features, since the
  high-level features are being learned from scratch, and you still want to take
  advantage of the features learned in the previous domain. This is definitely
  more of a heuristic than a developed theory, though.

\item For $k$-nearest neighbors, larger $k$ reduces variance but increases bias
  -- it controls model complexity. In the nearest neighbors metalearner, we
  aren't using nearest neighbors direction, but some smoothed-out version of it.
  How might you control model complexity for this alternative version of nearest
  neighbors?

  A: You could try to temper the distances that go in the exponent. This would
  let you have faster or slower decays of the smooth neighbor assignment
  function.

\item How would you adapt the ordinary classification-based nearest neighbors
  metalearner to work with continuous $y_i$ instead?

  A: You could try to learn a nearest neighbor regression.

\end{enumerate}

\section{Bayesian Deep Learning}

\begin{enumerate}

\item Assignments in mixture of Gaussians. Suppose $x_i$ is drawn from a mixture
  of two gaussians, which have parameters $\left(\mu_{1}, \sigma_{1}^2\right) =
  \left(0, 1\right)$ and $\left(\mu_{2}, \sigma_{2}^{2}\right) = \left(2,
  1\right)$. Show that $p\left(z = 1\vert x = 1\right) = \frac{1}{2}$ and
  $p\left(z = 1 \vert x = 0\right) = \frac{1}{1 + \exp{-2}} \approx 0.881$. In
  general the posterior is Bernoulli with probability $\varphi\left(x\right)$ of
  assigning to class 1. Can you find a formula for $\varphi\left(x\right)$ that
  applies to general (or multivariate?) $\mu_{k}, \Sigma_{k}$?

  A: By Bayes' rule,

  \begin{align*}
    p\left(z \vert x\right) &\propto p\left(x \vert z\right)p\left(z\right)\\
    &= \left(\Gsn\left(x \vert \mu_1, \Sigma_1\right)\right)^{\indic{z = 0}}
    \left(\Gsn\left(x \vert \mu_2, \Sigma_2\right)\right)^{\indic{z = 1}} \\
    &= \left(\frac{\Gsn\left(x \vert \mu_1, \Sigma_1\right)}{\Gsn\left(x \vert \mu_1, \Sigma_1\right) + \Gsn\left(x \vert \mu_1, \Sigma_1\right)} \right)^{\indic{z = 0}}
    \left(\frac{\Gsn\left(x \vert \mu_2, \Sigma_2\right)}{\Gsn\left(x \vert \mu_1, \Sigma_1\right) + \Gsn\left(x \vert \mu_1, \Sigma_1\right)} \right)^{\indic{z = 1}}
    %% &= \left(\Gsn\left(x \vert \mu_1, \Sigma_1\right) + \Gsn\left(x \vert \mu_2,
    %%   \Sigma_2\right)\right)\left(\frac{\Gsn\left(x \vert \mu_2,
    %%     \Sigma_2\right)}{\Gsn\left(x \vert \mu_1, \Sigma_1\right) + \Gsn\left(x
    %%     \vert \mu_1, \Sigma_1\right)\right)^{\indic{z = 0}}
  \end{align*}
  where in the last line we multiplied by the sum in each of the demoninators
  (which we can do because it doesn't depend on $z$, and we are working with
  something proportional to the density we want in the first place). Since the
  terms in parenthesis can be interpreted as $\pi$ and $1 - \pi$ for a bernoulli
  density, this gives us the final form of the conditional probability.
  Evaluating at the parameters in the problem gives the concrete numbers.

%% \item Jensen's inequality. The proof that $D_{KL}$ is nonnegative hinges heavily
%%   on a fact called Jensen's inequality. We prove this inequality here.
%% \begin{itemize}
%% \item A convex function is one for which, for any $\lambda \left(0, 1\right)$,
%%   we have $f\left(\lambda x + \left(1 - \lambda\right) y\right) \leq \lambda
%%   f\left(x\right) + \left(1 - \lambda\right)f\left(y\right)$. Interpret this
%%   inequality geometrically. (hint: consider the case of $\lambda = \frac{1}{2}$.
%% \item Let $f\left(x\right) = x^2$, and suppose $X \sim \Unif\left(-1, 1\right)$.
%% \item Hence, argue that $D_{KL}\left(q \vert \vert p\right)$
%% \end{itemize}

\item More general reparameterization. We saw that the Gaussian distribution
$\Gsn\left(x \vert \mu, \sigma^2 I\right)$ can be reparameterized as $g_{\mu,
  \sigma}\left(x\right) = \mu + \sigma \bigodot \eps$, where $\eps \sim
\Gsn\left(\eps \vert 0, I\right)$ doesn't depend on any parameters. This trick
actually applies for a variety of other distributions, which this exercise
explores.
\begin{itemize}
\item Random Variable generation using inverse CDFs\footnote{Cumulative Distribution Functions}. Suppose
  that $Z$, which we assume is one-dimensional, has CDF $F\left(z\right)$. Let
  $U \sim \Unif\left(0, 1\right)$. Verify that the transformation of $u$ defined
  by $F^{-1}\left(U\right)$ has CDF $F\left(z\right)$, and so has the same
  distribution as $Z$.

  A: Consider the probability that this new variable is less than some constant
  $z$,

  \begin{align*}
    \Parg{F^{-1}\left(U\right) \leq z} &= \Parg{U \leq F\left(z\right)} \\
    &= F\left(z\right)
  \end{align*}
  which is exactly the CDF of $Z$.
\item Argue that whenever the CDF of the density $q_{\varphi}\left(z \vert
  x\right)$ is known, this allows for a version of the reparameterization trick.
  The density of the uniform variable doesn't depend on any parameters. So you
  would just invert the deterministic CDF $F_{\varphi}$ and apply it to $U$,
  which serves the role of $\eps$ in the original VAE.

\item Suppose that $Z \sim \Exp{\lambda}$, meaning that it has CDF function
  $F\left(z\right) = 1 - \exp{-\lambda z}$. How can you simulate this?
  
  Using the inverse CDF transformation, part (a) tells us that we can just
  sample $-\frac{1}{\lambda}\log \left(1 - U\right)$ for uniform $U$.

\item Can you think of downsides of this approach?
  For many densities $q_{\varphi}$ of interest, the inverse CDF will usually not
  be available in closed form. This wouldn't be such a problem (the Gaussian CDF
  isn't available analytically, but we can evaluate it all the time) if it
  weren't for the fact that numerically inverting $F$ is often hard, and there
  are usually fewer methods for evaluating inverse CDFs than there are of
  ordinary CDFs.

\end{itemize}

\item Amortization vs. Approximation gaps. Recall that in the derivation of the
  ELBO, we had an expression like
  \begin{align*}
    \log p_{\theta}\left(x\right) &= \Esubarg{q}{\log p_{\theta}\left(x \vert
      z\right)} - D_{KL}\left(q\left(z \vert x\right) \vert \vert
    p\left(z\right)\right) + D_{KL}\left(q\left(z \vert x\right) \vert \vert
    p\left(z \vert x\right)\right)
  \end{align*}
  and we dropped the last term from the optimization, because it is intractable.
  We now study the role of that term when proposing variational families.

  \begin{itemize}
  \item In the usual VAE, we set $q_{\varphi}\left(z \vert x\right) =
    \Gsn\left(z \vert \mu_{\varphi}\left(x\right),
    \sigma_{\varphi}^{2}\left(x\right)I\right)$; i.e., a diagonal gaussian..
    Suppose you had approximated it instead by a Gaussian with general
    $\Sigma\left(x\right)$. What effect would this have on the approximation gap
    $D_{KL}\left(q_{\varphi^{\ast}}\left(z \vert x\right) \vert \vert p\left(z
    \vert x\right)\right)$, when considering the best possible
    $q_{\varphi^\ast}$ from either of these two (diagonal or dense covariance)
    variational families?

    The approximation gap would go down, since in theory the variational
    approximating family is larger when we include all possible covariances.

  \item Let $\hat{\varphi}$ be the parameters of the fitted inference network
    after optimizing the ELBO. Express the final inference quality
    $D_{KL}\left(q_{\hat{\varphi}}\left(z \vert x\right) \vert \vert p\left(z
    \vert x\right)\right)$ as,
    \begin{align*}
      D_{KL}\left(q_{\varphi^{\ast}}\left(z \vert x\right) \vert \vert p\left(z
      \vert x\right)\right) + \left(D_{KL}\left(q_{\hat{\varphi}}\left(z \vert
      \vert x\right)\right) - D_{KL}\left(q_{\varphi^{\ast}}\left(z \vert x
        \right)\right)\right).
    \end{align*}
    The first term (outside parenthesis) is called the ``approximation gap,''
    and refers to the difference between the true posterior and the best
    possible element of the variational approximation, while the second term (in
    parenthesis) is called the ``amortization gap,'' and refers to the
    difference between the best possible approximation within the family and
    what is actually found by the network. In light of this discussion, why
    might we choose not to proceed with the full $\Sigma\left(x\right)$
    parameterization in the previous part?

    We risk having a very large amortization gap, since it might be much harder
    to learn a good network that maps individual datapoints to these full
    covariance matrices.

  \item A normalizing flow is a sequence of transformations to a simple variable
    that results in a variable with a more complicated density, but one which
    can still be written in closed form, using the change of variables formula.
    For example, you might iteratively apply $f\left(z\right) = z +
    u\sigma\left(w^T z + b\right)$ to what is intiially a simple (say gaussian
    $z$), since this transformation is easy to differentiate (which is the only
    thing you need to apply the change of variables formula). Does this proposed
    procedure reduce the approximation or amortization gap?

    This reduces the approximation gap, because it creates a more expressive
    variational family (not just diagonal gaussians anymore).

  \item What are some general strategies for reducing the amortization gap?

    Usually people just train larger encoder networks, hoping that their larger capacity 
    will help you find $q_{\varphi}\left(z \vert x\right)$ that is close to the
    best one from the variational family.
  \end{itemize}
\end{enumerate}

\end{document}
